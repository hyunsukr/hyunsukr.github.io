%---------change this every homework
\def\yourid{Hyun Suk Ryoo}
\def\collabs{}
% -----------------------------------------------------
\def\duedate{10/4/18 11:59PM}
\def\duelocation{via Collab}
\def\prof{hr2ee}
\def\course{{Math 3100 (Introduction to Probability)}}%------
%-------------------------------------
%-------------------------------------

\documentclass[10pt]{report}
\usepackage[colorlinks,urlcolor=blue]{hyperref}
\usepackage[osf]{mathpazo}
\usepackage{amsmath,amsfonts,graphicx}
\usepackage{latexsym}
\usepackage[shortlabels]{enumitem}
\usepackage[top=1in,bottom=1.4in,left=1.5in,right=1.5in,centering]{geometry}
\usepackage{color}
\definecolor{mdb}{rgb}{0.3,0.02,0.02} 
\definecolor{cit}{rgb}{0.05,0.2,0.45} 
\pagestyle{myheadings}
\markboth{\yourid}{\yourid}
\usepackage{clrscode}
\usepackage{url}

\newenvironment{proof}{\par\noindent{\it Proof.}\hspace*{1em}}{$\Box$\bigskip}
\newcommand{\handout}{
   \renewcommand{\thepage}{}
   \noindent
   \begin{center}
      \vbox{
    \hbox to \columnwidth {\sc{\course} --- \prof \hfill}
    \vspace{-2mm}
    \hbox to \columnwidth {\sc due \MakeLowercase{\duedate} \duelocation \hfill {\LARGE\color{mdb}\yourid}}
      }
   \end{center}
      Homework \# 5
   \vspace*{2mm}
}
\newcommand{\solution}[1]{\medskip\noindent\textbf{Solution:}#1}
\newcommand{\bit}[1]{\{0,1\}^{ #1 }}
%\dontprintsemicolon
%\linesnumbered
\newtheorem{problem}{\sc\color{cit}section}
\newtheorem{practice}{\sc\color{cit}practice}
\newtheorem{lemma}{Lemma}
\newtheorem{definition}{Definition}

\def\therefore{\boldsymbol{\text{ }
\leavevmode
\lower0.4ex\hbox{$\cdot$}
\kern-.5em\raise0.7ex\hbox{$\cdot$}
\kern-0.6em\lower0.4ex\hbox{$\cdot$}
\thinspace\text{ }}}

\begin{document}
\thispagestyle{empty}
\handout
%----Begin your modifications here

\setcounter{chapter}{3} 
\setcounter{section}{0}
\section{\sc\color{cit}Random Variables Introduction}
\setcounter{subsection}{1}
\subsection{}
 \begin{enumerate}[(a)]
        \item \ \\
        \begin{center}
        Joint Distribution for sampling with replacement...
 \begin{tabular}{|c| c|c |c |c|} 
 \hline
  & $x=1$ & $x=2$ & $x=3$ & $x=4$ \\
 \hline
 $y=1$ & $\frac{1}{16}$ & $\frac{1}{16}$& $\frac{1}{16}$& $\frac{1}{16}$\\ 
 \hline
 $y=2$ & $\frac{1}{16}$& $\frac{1}{16}$& $\frac{1}{16}$& $\frac{1}{16}$\\
 \hline
 $y=3$& $\frac{1}{16}$& $\frac{1}{16}$& $\frac{1}{16}$& $\frac{1}{16}$ \\
 \hline
 $y=4$ & $\frac{1}{16}$& $\frac{1}{16}$& $\frac{1}{16}$& $\frac{1}{16}$ \\
 \hline
\end{tabular}\ \\
	There are $4 $ cases where $1 $ is less than or equal to the $y$ values, $3 $ cases for $2 $, $2 $ for $3$ and $1 $ for $4$. This means the possibilities are $4 + 3 + 2 + 1$\\
 $\therefore P(x \leq y) = \frac{10}{16} $
\end{center}
        \item \ \\   
                \begin{center}
        Joint Distribution for sampling without replacement...
 \begin{tabular}{|c| c|c |c |c|} 
 \hline
  & $x=1$ & $x=2$ & $x=3$ & $x=4$ \\
 \hline
 $y=1$ & $N/A$ & $\frac{1}{16}$& $\frac{1}{16}$& $\frac{1}{16}$\\ 
 \hline
 $y=2$ & $\frac{1}{16}$& $N/A$& $\frac{1}{16}$& $\frac{1}{16}$\\
 \hline
 $y=3$& $\frac{1}{16}$& $\frac{1}{16}$& $N/A$& $\frac{1}{16}$ \\
 \hline
 $y=4$ & $\frac{1}{16}$& $\frac{1}{16}$& $\frac{1}{16}$& $N/A$ \\
 \hline
\end{tabular}\ \\
	There are $3 $ cases where $1 $ is less than or equal to the $y$ values, $2$ cases for $2 $, $1$ for $3$ and $0 $ for $4$. This means the possibilities are $3 + 2 + 1 + 0$\\
 $\therefore P(x \leq y) = \frac{6}{12} $
 \end{center}
    \end{enumerate}
    \subsection{}
 \begin{enumerate}[(a)]
 \item The range of the sum of rolling a die twice is between $2 $ and $12 $. This is because the lowest number possible will be $1 $ occurring two times, and the highest number possible will be $6$ occurring two times.
 \item \ \\
 \begin{center}
        Distribution of S...\\
 \begin{tabular}{|c| c|c |c |c|c|c|c|c|c|c|c|} 
 \hline
 S & $2$& $3$ & $4$  & $5$ &$6$ &$7$ & $8$ & $9$ & $10 $& $11 $& $12$ \\
 \hline
 Probability & $\frac{1}{36}$  &$\frac{2}{36}$  &  $\frac{3}{36}$ &  $\frac{4}{36}$& $\frac{5}{36}$& $\frac{6}{36}$&  $\frac{5}{36}$&  $\frac{4}{36}$&$\frac{3}{36}$  & $\frac{2}{36}$ & $\frac{1}{36}$ \\
 \hline
 Possibilities & $1,1 $ & $1,2$ & $1,3$  & $1,4$&$1,5$ &$1,6$ & $2,6$ & $3,6$ & $4,6$ & $5,6$ & $6,6$ \\
  &  & $2,1$ & $2,2$  &  $2,3$&$2,4$, &$2,7$& $3,5$ & $4,5$ & $5,5$ & $6,5$ &  \\  
  &  &  & $3,1$  & $3,2$ &$3,3$ &$3,4$ & $4,4$ & $5,4$ & $6,4$ &  &  \\  
  &  &  &   & $4,1$ &$4,2$ &$4,3$ & $5,3$ & $6,3$ &  &  &  \\  
  &  &  &   &  &$5,1$ &$5,2$ & $6,2$ &  &  & &  \\
    &  &  &   &  & &$6,1$ & & &  &  &  \\
 \hline
 \end{tabular}
 \end{center}
 \end{enumerate}
 \subsection{}
 \begin{enumerate}[(a)]
        \item \ \\
        \begin{center}
        Joint Distribution for $(X_1,X_2)$... \\
 \begin{tabular}{|c| c|c|c|c|c|c|} 
 \hline
  & $x_1=1$ & $x_2=2$ & $x_3=3$ & $x_4=4$ & $x_5=5$ & $x_6=6$ \\
 \hline
 $x_2=1$ & $\frac{1}{36}$ & $\frac{1}{36}$& $\frac{1}{36}$& $\frac{1}{36}$& $\frac{1}{36}$& $\frac{1}{36}$\\ 
 \hline
 $x_2=2$ & $\frac{1}{36}$& $\frac{1}{36}$& $\frac{1}{36}$& $\frac{1}{36}$& $\frac{1}{36}$& $\frac{1}{36}$\\
 \hline
 $x_2=3$& $\frac{1}{36}$& $\frac{1}{36}$& $\frac{1}{36}$& $\frac{1}{36}$& $\frac{1}{36}$& $\frac{1}{36}$ \\
 \hline
 $x_2=4$ & $\frac{1}{36}$& $\frac{1}{36}$& $\frac{1}{36}$& $\frac{1}{36}$& $\frac{1}{36}$& $\frac{1}{36}$ \\
 \hline
  $x_2=5$ & $\frac{1}{36}$& $\frac{1}{36}$& $\frac{1}{36}$& $\frac{1}{36}$& $\frac{1}{36}$& $\frac{1}{36}$\\
 \hline
  $x_2=6$ & $\frac{1}{36}$& $\frac{1}{36}$& $\frac{1}{36}$& $\frac{1}{36}$& $\frac{1}{36}$& $\frac{1}{36}$ \\
 \hline
\end{tabular}\ \\
\end{center}
        \item \ \\
        \begin{center}
        Joint Distribution for $(Y_1,Y_2)$... \\
 \begin{tabular}{|c|c|c|c|c|c|c|} 
 \hline
  & $y_1=1$ & $y_2=2$ & $y_3=3$ & $y_4=4$ & $y_5=5$ & $y_6=6$ \\
 \hline
 $y_2=1$ & $\frac{1}{36}$ & $\frac{2}{36}$& $\frac{2}{36}$& $\frac{2}{36}$& $\frac{2}{36}$& $\frac{2}{36}$\\ 
 \hline
 $y_2=2$ & $0$& $\frac{1}{36}$& $\frac{2}{36}$& $\frac{2}{36}$& $\frac{2}{36}$& $\frac{2}{36}$\\
 \hline
 $y_2=3$& $0$& $0$& $\frac{1}{36}$& $\frac{2}{36}$& $\frac{2}{36}$& $\frac{2}{36}$ \\
 \hline
 $y_2=4$ & $0$& $0$& $0$& $\frac{1}{36}$& $\frac{2}{36}$& $\frac{2}{36}$ \\
 \hline
  $y_2=5$ & $0$& $0$& $0$& $0$& $\frac{1}{36}$& $\frac{2}{36}$\\
 \hline
  $y_2=6$ & $0$& $0$& $0$& $0$& $0$& $\frac{1}{36}$ \\
 \hline
\end{tabular}\ \\
\end{center}
    \end{enumerate}
    \setcounter{subsection}{14}
    \subsection{}
 \begin{enumerate}[(a)]
 \item This is problem c...
 \begin{proof}
 Looking at the possibilities of this probability, we can see that it is a simple summation problem. For when $X=1$ there is no way that this possibility will hold. However for $X=2$ there will be $1 $ way, which is $Y=1$. This patter of $ways = x-1$ will equate to be $1 + 2 + 3 ... n-1 $. The number of conditions that satisfy this rule over the total ($n^2$) will be the following.
 \begin{align*}
 &\frac{1+2+3...+n-1}{n^2} \\
 &\frac{\frac{(n-1)*n}{2}}{n^2}\\
 &\frac{n-1}{2n}\\
 &\therefore P= \mathbf{\frac{n-1}{2n}}
 \end{align*}
 \end{proof}
 \item Problem e...
 \begin{proof} \ \\
 We can start by looking at probabilities for when $k=$ a certain number.
 \begin{align*}
 k = 1 &= (1,1) (1,2) ... (1,n) ... \\
 &= (2,1) (3,1) ... (n,1) \\
 & = n + n -1 \\
 k=2 &= (2,2) (2,3) ...(2,n)...\\
 &= (3,2) (4,2) .... (n, 2) \\
 &= n-1 + n -2 \\
 k= n -1 &= (n-1, n-1) (n-1, n) \\
 &= (n, n-1) \\
 &= 3 \\
 k = n &= (n,n) \\
 &= 1
 \end{align*}
 \begin{center}
 The general rule is $2(n-k+1)-1 $.
 \end{center}
 The number of possibilities is $n^2$. Therefore, the probability will then become $\mathbf{\frac{2(n-k+1)-1 }{n^2}}$ \\
 \end{proof}
 \item Problem f ...
 \begin{proof}\\
 Let us visualize. 
 \begin{align*}
 k &= 1 = N/A = 0\\
 k&= 2 = (1,1) = 1 \\
 k&= 3 = (1,2), (2,1) = 2 \\
 k&= n-1 = ... = k - 1 -1 \\
 k&= n = ... = k - 1 \\
 \end{align*}
  \begin{center}
 The general rule is $k-1 $.
 \end{center}
 The number of possibilities is $n^2$. Therefore, the probability will then become $\mathbf{\frac{k-1}{n^2}}$ \\
 \end{proof}
 \end{enumerate}
     \setcounter{subsection}{22}
    \subsection{}
 \begin{enumerate}[(a)]
 \item \begin{proof}

The condition of $x \leq t $ will be a subset of the condition $y \leq t $ because of the condition of $ x \geq y $. We can say that $\{(x,y) | x \leq t \} \subseteq \{(x,y) | y\leq t \} $. This means that $\mathbf{P(x\leq t) \leq P(y \leq t)} $ Since the number of events occurring is greater for the second condition.\\
 \end{proof}
 \end{enumerate}
        \setcounter{section}{1}
 \section{\sc\color{cit}Expectations}
      \setcounter{subsection}{2}
    \subsection{}
 \begin{enumerate}[(a)]
 \item The total outcome space has $6*6*6 = 216$ combinations. There are $6 $ possibilities for all three places of $(a,b,c) $. \ \\
 Let's look at the possibilities of sixes. ...
 \begin{align*}
 P(3 six) & = \frac{1}{216} \\
 &= (6,6,6) \\
 P(2 six) &= \frac{15}{216} \\
 &= ((5 \textnormal{ possibilities}),6 , 6), (6, (5\textnormal{ possibilities}), 6), (6, 6, (5\textnormal{ possibilities})) \\
 P(1 six) &= \frac{75}{216} \\
 &= (6, 1, (5\textnormal{ possibilities})) ... (6, 5, (5\textnormal{ possibilities})) (6, (5\textnormal{ possibilities}), 1) ... (6,(5\textnormal{ possibliities}), 5)... \\
 P(0 six) &= \frac{125}{216} \\
 &= 5 * 5 * 5 \\
 Expectation &= 0*\frac{125}{216} + 1*\frac{75}{216} + 2*\frac{15}{216} + 3*\frac{1}{216} \\
 &=\mathbf{ \frac{1}{2}}
 \end{align*}
  \item The total outcome space has $6*6*6 = 216$ combinations. There are $6 $ possibilities for all three places of $(a,b,c) $. \ \\
 Let's look at the possibilities of sixes. ...
 \begin{align*}
 P(\textnormal{3 odd}) &= \frac{27}{216} \\
 P(\textnormal{2 odd}) &= \frac{81}{216} \\
 P(\textnormal{1 odd}) &= \frac{81}{216} \\
 P(\textnormal{0 odd}) &= \frac{27}{216} \\
Expectation &= 0 * \frac{27}{216} + 1 * \frac{81}{216} + 2 *\frac{81}{216} + 3 * \frac{27}{216} \\
&= \mathbf{1.5}
\end{align*}
 \end{enumerate}
 \subsection{}
 \begin{enumerate}[(a)]
 \item For this problem we will be able to use the Markov's Inequality...
 \begin{align*}
 If X \geq 0, then P(X \geq a)& \leq \frac{E(X)}{a} \textnormal{ for every } a > 0 \\
 P(X > 8) &\leq \frac{2}{8} \\
 P(X> 8) & \leq \frac{1}{4} \\
 \end{align*}
 We can conclude that at least $\frac{1}{4} $ of the 100 numbers is 25 and at least are greater than 8.
 \end{enumerate}
\end{document}
