%---------change this every homework
\def\yourid{Hyun Suk Ryoo}
\def\collabs{}
% -----------------------------------------------------
\def\duedate{9/24/18 11:59PM}
\def\duelocation{via Collab}
\def\prof{hr2ee}
\def\course{{Math 3100 (Introduction to Probability)}}%------
%-------------------------------------
%-------------------------------------

\documentclass[10pt]{report}
\usepackage[colorlinks,urlcolor=blue]{hyperref}
\usepackage[osf]{mathpazo}
\usepackage{amsmath,amsfonts,graphicx}
\usepackage{latexsym}
\usepackage[shortlabels]{enumitem}
\usepackage[top=1in,bottom=1.4in,left=1.5in,right=1.5in,centering]{geometry}
\usepackage{color}
\definecolor{mdb}{rgb}{0.3,0.02,0.02} 
\definecolor{cit}{rgb}{0.05,0.2,0.45} 
\pagestyle{myheadings}
\markboth{\yourid}{\yourid}
\usepackage{clrscode}
\usepackage{url}

\newenvironment{proof}{\par\noindent{\it Proof.}\hspace*{1em}}{$\Box$\bigskip}
\newcommand{\handout}{
   \renewcommand{\thepage}{}
   \noindent
   \begin{center}
      \vbox{
    \hbox to \columnwidth {\sc{\course} --- \prof \hfill}
    \vspace{-2mm}
    \hbox to \columnwidth {\sc due \MakeLowercase{\duedate} \duelocation \hfill {\LARGE\color{mdb}\yourid}}
      }
   \end{center}
      Homework \# 4
   \vspace*{2mm}
}
\newcommand{\solution}[1]{\medskip\noindent\textbf{Solution:}#1}
\newcommand{\bit}[1]{\{0,1\}^{ #1 }}
%\dontprintsemicolon
%\linesnumbered
\newtheorem{problem}{\sc\color{cit}section}
\newtheorem{practice}{\sc\color{cit}practice}
\newtheorem{lemma}{Lemma}
\newtheorem{definition}{Definition}

\def\therefore{\boldsymbol{\text{ }
\leavevmode
\lower0.4ex\hbox{$\cdot$}
\kern-.5em\raise0.7ex\hbox{$\cdot$}
\kern-0.6em\lower0.4ex\hbox{$\cdot$}
\thinspace\text{ }}}

\begin{document}
\thispagestyle{empty}
\handout
%----Begin your modifications here

\setcounter{chapter}{2} 
\setcounter{section}{0}
\section{\sc\color{cit}The Binomial Distribution}
\setcounter{subsection}{11}
\subsection{}
 \begin{enumerate}[(a)]
        \item We can assume from previous problems and page 7 that the probability of winning is $\frac{18}{38} $. This problem asks for what is the probability that she has to make exactly 8 bets before stopping. This means that what is the probability that she won exactly $\frac{4}{7} $ and wins on her $8^{th} $ trial.
        \begin{proof}\ \\
        $A $ = above situation ...
        \begin{align*}
        P(A) &= \binom{7}{4} * \bigg(\frac{18}{38}\bigg)^4 * \bigg(\frac{20}{38}\bigg)^3 * \frac{18}{38} \\
        &= \bigg(\frac{7!}{4!* 3!}\bigg) * \bigg(\frac{18}{38}\bigg)^4 * \bigg(\frac{20}{38}\bigg)^3 * \frac{18}{38} \\
         &= \bigg(\frac{7*6*5}{3*2}\bigg) * \bigg(\frac{18}{38}\bigg)^4 * \bigg(\frac{20}{38}\bigg)^3 * \frac{18}{38} \\
         &=0.121689 \\
         \therefore P(A) &= \mathbf{0.122}
        \end{align*}
        \end{proof}
        \item The condition that she must at least play 9 times implies that she hasn't one any, won once, twice, three times, or four times. We must calculate these probabilities...
        \begin{proof}
        $A $ = above situation....
        \begin{align*}
        P(A) &= \bigg(\frac{20}{38}\bigg) ^ {8} + \binom{8}{1} * \bigg(\frac{20}{38}\bigg)^7 * \bigg(\frac{18}{38}\bigg) \\&+ \binom{8}{2} * \bigg(\frac{20}{38}\bigg)^6 * \bigg(\frac{18}{38}\bigg)^2 +\binom{8}{3} * \bigg(\frac{20}{38}\bigg)^6 * \bigg(\frac{18}{38}\bigg)^3 + \binom{8}{4} * \bigg(\frac{20}{38}\bigg)^4 * \bigg(\frac{18}{38}\bigg)^4 \\
        &= \bigg(\frac{20}{38}\bigg) ^ {8} +\frac{8}{1} * \bigg(\frac{20}{38}\bigg)^7 * \bigg(\frac{18}{38}\bigg) \\&+ \frac{8*7}{2} * \bigg(\frac{20}{38}\bigg)^6 * \bigg(\frac{18}{38}\bigg)^2 +\frac{8*7*6}{3*2} * \bigg(\frac{20}{38}\bigg)^5 * \bigg(\frac{18}{38}\bigg)^3 + \frac{8*7*6*5}{4*3*2} * \bigg(\frac{20}{38}\bigg)^4 * \bigg(\frac{18}{38}\bigg)^4 \\
        &= 0.6926 \\
        \therefore P(A) &= \mathbf{0.693}
        \end{align*}
        \end{proof}
    \end{enumerate}
    \setcounter{subsection}{6}
        \setcounter{section}{0}
 \section{\sc\color{cit}An extra problem on the binomial distribution}
Toss a fair coin 100 times. The expected number of Heads is 50. Use a calculator to compute the probability that there are exactly 50 heads. Discuss the answer. Does it feel intuitive?
\ \\
\begin{proof}
We can solve this with binomial distribution...
\begin{align*}
P(Event) &= \binom{100}{50} * \bigg(\frac{1}{2}\bigg)^{50} * \bigg(\frac{1}{2}\bigg)^{50} \\
 &= \binom{100}{50} * \bigg(\frac{1}{2^{100}}\bigg) \\
\end{align*}
We can use normal approximation to find this number.
\begin{align*}
\mu &= n*p = 100*\frac{1}{2} = 50\\
\sigma &= \sqrt{np*(1-p)} = \sqrt{100*\frac{1}{4}} = 5\\
P(a \leq \#H \leq b) &\simeq \phi \bigg(\frac{b+\frac{1}{2} - \mu}{\sigma}\bigg) - \phi \bigg(\frac{a-\frac{1}{2} - \mu}{\sigma}\bigg) \\
P(50 \leq \#H \leq 50) &\simeq \phi \bigg(\frac{50+\frac{1}{2} - 50}{5}\bigg) - \phi \bigg(\frac{50-\frac{1}{2} - 50}{5}\bigg) \\
P(50 \leq \#H \leq 50) &\simeq \phi (0.1) - \phi(-0.1) \\
P(50 \leq \#H \leq 50) &\simeq \phi (0.1) - (1-\phi(0.1)) \\
P(50 \leq \#H \leq 50) &\simeq 0.5398 - (1-(0.5398)) \\
P(50 \leq \#H \leq 50) &\simeq 0.0796 \\
\therefore P(\textnormal{Exactly 50 heads}) &= \mathbf{0.08} \\
\end{align*}
This does not seem intuitive since the probability of getting heads is $\frac{1}{2} $, you would assume around $50 $ heads would show up. It seems intuitive that $P(50 Heads) $ will be the most common, but it is actually only $\mathbf{8\%} $ \\
\end{proof}
\section{\sc\color{cit}Normal Approximation Method}
\textbf{ALL Z SCORE VALUES WERE TAKEN TO THE NEAREST $\mathbf{0.01} $}
\setcounter{subsection}{0}
 \subsection{}
 \begin{enumerate}[(a)]
\item $400 $ tosses of a fair coin...
\begin{align*}
\mu &= n*p = 400*\frac{1}{2} = 200\\
\sigma &= \sqrt{np*(1-p)} = \sqrt{400*\frac{1}{4}} = 10\\
P(a \leq \#H \leq b) &\simeq \phi \bigg(\frac{b+\frac{1}{2} - \mu}{\sigma}\bigg) - \phi \bigg(\frac{a-\frac{1}{2} - \mu}{\sigma}\bigg) \\
P(190 \leq \#H \leq 210) &\simeq \phi \bigg(\frac{210+\frac{1}{2} - 200}{10}\bigg) - \phi \bigg(\frac{190-\frac{1}{2} - 200}{10}\bigg) \\
P(190 \leq \#H \leq 210) &\simeq \phi \bigg(\frac{10.5}{10}\bigg) - \phi \bigg(\frac{-10.5}{10}\bigg) \\
P(190 \leq \#H \leq 210) &\simeq \phi (1.05) - \phi(-1.05) \\
P(190 \leq \#H \leq 210) &\simeq \phi (1.05) - (1-\phi(1.05)) \\
P(190 \leq \#H \leq 210) &\simeq 0.85314 - (1-(0.85314)) \\
P(190 \leq \#H \leq 210) &\simeq 0.70628 \\
\therefore P(190 \leq \#H \leq 210) &= \mathbf{0.7062} \\
\end{align*}
\item Problem \textbf{C} Sorry for the wrong numbering...
\begin{align*}
\mu &= n*p = 400*\frac{1}{2} = 200\\
\sigma &= \sqrt{np*(1-p)} = \sqrt{400*\frac{1}{4}} = 10\\
P(a \leq \#H \leq b) &\simeq \phi \bigg(\frac{b+\frac{1}{2} - \mu}{\sigma}\bigg) - \phi \bigg(\frac{a-\frac{1}{2} - \mu}{\sigma}\bigg) \\
P(200 \leq \#H \leq 200) &\simeq \phi \bigg(\frac{200+\frac{1}{2} - 200}{10}\bigg) - \phi \bigg(\frac{200-\frac{1}{2} - 200}{10}\bigg) \\
P(200 \leq \#H \leq 200) &\simeq \phi \bigg(\frac{0.05}{10}\bigg) - \phi \bigg(\frac{-0.05}{10}\bigg) \\
P(200 \leq \#H \leq 200) &\simeq \phi (0.05) - \phi(-0.05) \\
P(200 \leq \#H \leq 200) &\simeq \phi (0.05) - (1-\phi(0.05)) \\
P(200 \leq \#H \leq 200) &\simeq 0.51994 - (1-(0.51994)) \\
P(200 \leq \#H \leq 200) &\simeq 0.03988 \\
\therefore P(\#H = 200) &= \mathbf{0.03988} \\
\end{align*}
\item Problem \textbf{D} Sorry for the wrong numbering...
\begin{align*}
\mu &= n*p = 400*\frac{1}{2} = 200\\
\sigma &= \sqrt{np*(1-p)} = \sqrt{400*\frac{1}{4}} = 10\\
P(a \leq \#H \leq b) &\simeq \phi \bigg(\frac{b+\frac{1}{2} - \mu}{\sigma}\bigg) - \phi \bigg(\frac{a-\frac{1}{2} - \mu}{\sigma}\bigg) \\
P(210 \leq \#H \leq 210) &\simeq \phi \bigg(\frac{210+\frac{1}{2} - 200}{10}\bigg) - \phi \bigg(\frac{210-\frac{1}{2} - 200}{10}\bigg) \\
P(210 \leq \#H \leq 210) &\simeq \phi \bigg(\frac{10.5}{10}\bigg) - \phi \bigg(\frac{9.5}{10}\bigg) \\
P(200 \leq \#H \leq 200) &\simeq \phi (1.05) - \phi(0.95) \\
P(200 \leq \#H \leq 200) &\simeq 0.85314 - 0.82894 \\
P(200 \leq \#H \leq 200) &\simeq 0.0242 \\
\therefore P(\#H = 200) &= \mathbf{0.0242} \\
\end{align*}
 \end{enumerate}
  \subsection{}
 \begin{enumerate}[(a)]
\item $400 $ tosses of a $P(heads)=0.51 $ coin...
\begin{align*}
\mu &= n*p = 400*0.51 = 204\\
\sigma &= \sqrt{np*(1-p)} = \sqrt{400*0.51*0.49} = 9.998\\
P(a \leq \#H \leq b) &\simeq \phi \bigg(\frac{b+\frac{1}{2} - \mu}{\sigma}\bigg) - \phi \bigg(\frac{a-\frac{1}{2} - \mu}{\sigma}\bigg) \\
P(190 \leq \#H \leq 210) &\simeq \phi \bigg(\frac{210+\frac{1}{2} - 204}{9.998}\bigg) - \phi \bigg(\frac{190-\frac{1}{2} - 204}{9.998}\bigg) \\
P(190 \leq \#H \leq 210) &\simeq \phi (0.6501) - \phi(-1.4502) \\
P(190 \leq \#H \leq 210) &\simeq \phi (0.6501) - (1-\phi(1.4502)) \\
P(190 \leq \#H \leq 210) &\simeq 0.74215 - (1-(0.92647)) \\
P(190 \leq \#H \leq 210) &\simeq 0.66862 \\
\therefore P(190 \leq \#H \leq 210) &= \mathbf{0.66862} \\
\end{align*}
\item Problem \textbf{C} Sorry for the wrong numbering...
\begin{align*}
\mu &= n*p = 400*0.51 = 204\\
\sigma &= \sqrt{np*(1-p)} = \sqrt{400*0.51*0.49} = 9.998\\
P(a \leq \#H \leq b) &\simeq \phi \bigg(\frac{b+\frac{1}{2} - \mu}{\sigma}\bigg) - \phi \bigg(\frac{a-\frac{1}{2} - \mu}{\sigma}\bigg) \\
P(200 \leq \#H \leq 200) &\simeq \phi \bigg(\frac{200+\frac{1}{2} - 204}{9.998}\bigg) - \phi \bigg(\frac{200-\frac{1}{2} - 204}{9.998}\bigg) \\
P(200 \leq \#H \leq 200) &\simeq \phi (-0.3501) - \phi(-0.4501) \\
P(200 \leq \#H \leq 200) &\simeq (1- \phi (0.3501) - (1-\phi(0.4501)) \\
P(200 \leq \#H \leq 200) &\simeq (1-0.63683) - (1-0.67364) \\
P(200 \leq \#H \leq 200) &\simeq 0.03681 \\
\therefore P(\#H = 200) &= \mathbf{0.03681} \\
\end{align*}
\item Problem \textbf{D} Sorry for the wrong numbering...
\begin{align*}
\mu &= n*p = 400*0.51 = 204\\
\sigma &= \sqrt{np*(1-p)} = \sqrt{400*0.51*0.49} = 9.998\\
P(a \leq \#H \leq b) &\simeq \phi \bigg(\frac{b+\frac{1}{2} - \mu}{\sigma}\bigg) - \phi \bigg(\frac{a-\frac{1}{2} - \mu}{\sigma}\bigg) \\
P(210 \leq \#H \leq 210) &\simeq \phi \bigg(\frac{210+\frac{1}{2} - 204}{9.998}\bigg) - \phi \bigg(\frac{210-\frac{1}{2} - 204}{9.998}\bigg) \\
P(210 \leq \#H \leq 210) &\simeq \phi (0.6501) - \phi(0.55011) \\
P(210 \leq \#H \leq 210) &\simeq (0.74215) - (0.70884) \\
P(210 \leq \#H \leq 210) &\simeq 0.03331 \\
\therefore P(\#H = 210) &= \mathbf{0.03331} \\
\end{align*}
 \end{enumerate}
 \setcounter{subsection}{7}
 \subsection{}
 \begin{enumerate}[(a)]
 \item We would use normal approximation method for this as well...
 \begin{align*}
\mu &= n*p = 600*\frac{1}{6} = 100\\
\sigma &= \sqrt{np*(1-p)} = \sqrt{600*\frac{1}{6}*\frac{5}{6}} = 9.128709\\
P(a \leq \#H \leq b) &\simeq \phi \bigg(\frac{b+\frac{1}{2} - \mu}{\sigma}\bigg) - \phi \bigg(\frac{a-\frac{1}{2} - \mu}{\sigma}\bigg) \\
P(100 \leq \#H \leq 100) &\simeq \phi \bigg(\frac{100+\frac{1}{2} - 100}{9.128709}\bigg) - \phi \bigg(\frac{100-\frac{1}{2} - 100}{9.128709}\bigg) \\
P(100 \leq \#H \leq 100) &\simeq \phi (0.05477) - \phi(-0.05477) \\
P(100 \leq \#H \leq 100) &\simeq \phi (0.05477) - (1-\phi(0.05477)) \\
P(100 \leq \#H \leq 100) &\simeq (0.51994) - (1-0.51994) \\
P(100 \leq \#H \leq 100) &\simeq 0.03988 \\
\therefore P(\#6 = 100) &= \mathbf{0.03988} \\
\end{align*}
 \end{enumerate}
\end{document}
